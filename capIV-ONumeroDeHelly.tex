\chapter{O número de Helly e o número de Helly forte para grafos $B_k$-EPG e $B_k$-VPG}


\begin{flushright}
\begin{minipage}[t][0cm][b]{0.47\textwidth}
\emph{
%A Matemática é o alfabeto com o qual Deus escreveu o Universo.
Falta algo para completar esta demonstração, mas não tenho tempo.}
\end{minipage}

\rule[0cm]{7cm}{0.03cm}%{largura}{espessura}

Évariste Galois
\end{flushright}




O estudo de grafos EPG foi introduzido por  Golumbic, Lypshteyn e Stern (2009) e consiste dos grafos de intersecção de conjuntos de caminhos sobre uma grade ortogonal, cujas intersecções são tomadas considerando as arestas dos caminhos. Se as intersecções dos caminhos consideram os vértices e não as arestas, a classe de grafos resultante é chamada de grafos VPG. Tal classe foi introduzida em 2011 \cite{asinowski2011string} e \cite{asinowski2012}. Nesse capítulo estudaremos dois parâmetros em ambas classes de grafos EPG e VPG. Os parâmetros que serão estudados são nomeadamente o número de Helly e o número de Helly forte.

\section{Definições iniciais sobre número de Helly e número de Helly forte}

Seja  $\cal {F}$ uma família de conjuntos de algum conjunto universal $U$, e $h$ um número inteiro tal que $h\geq 1$. Podemos dizer que $\cal{F}$ é $h$-{\it intersectante} quando todos  $h$ subconjuntos de $\cal {F}$ intersectam-se. Chamamos de {\it core} de $\cal {F}$ a intersecção de todos conjuntos de $\cal {F}$, e denotamos por $core(\cal F)$. 

A família $\cal{F}$ é $h$-{\it Helly} quando toda subfamília $h$-intersectante $\cal{F'}$ satisfaz $core(\cal{F'}) \neq \emptyset$, ver mais em \cite{duchet1978propriete}. Por outro lado, se para toda subfamília $\cal{F'}$ de $\cal{F}$, existem $h$ subconjuntos cujo core é igual ao core de  $\cal {F'}$, então $\cal {F}$ é dito ser  $h$-{\it Helly} {\it forte}. Claramente, se $\cal {F}$ é $h$-Helly então ele também é $h'$-Helly, para $h' \geq h$. Similarmente, se ${\cal F}$ é $h$-Helly forte então ele também é $h'$-Helly forte, para $h' \geq h$. 

Finalmente, o   {\it número de Helly} da família  $\cal{F}$ é o menor inteiro $h$, tal que $\cal{F}$ é  $h$-Helly. Similarmente, o {\it número de Helly forte} de  $\cal{F}$ é o menor $h$, para o qual  $\cal{F}$ é  $h$-Helly forte. Também segue que o número de Helly forte de $\cal{F}$ é no mínimo igual ao seu número de Helly.


Uma  {\it classe} $\cal {C}$ de famílias $\cal {F}$  de subconjuntos de algum conjunto universal $U$ é uma  subcoleção das famílias $\cal {F}$ de $U$. Dizemos que  $\cal C$ é uma {\it classe hereditária}, quando ela é fechada sob inclusão, i.e. se um grafo $G$ pertence a uma classe $C$ então todo subgrafo induzido de $G$ também pertence a $C$. O {\it número de Helly}  de uma classe  $\cal{C}$ de famílias $\cal{F}$ de subconjuntos é o maior número de Helly entre todas as famílias de $\cal {F}$. Similarmente, o {\it número de Helly forte} de uma classe  $\cal {C}$ é o maior número de Helly forte das famílias de $\cal {C}$.

Se $\cal F$ é uma família de subconjuntos e $\cal C$ uma classe de famílias, denotamos por $H(\cal F)$ e por 
$H(\cal C)$,  o número de  Helly de $\cal F$ e $\cal C$, respectivamente, enquanto  $sH({\cal F})$ e $sH({\cal C})$  representam os números de  Helly forte de $\cal F$ e $\cal C$.


Nesse capítulo, nos preocupamos com famílias de subconjuntos $\cal{F}$ de caminhos de arestas e vértices em uma grade. No primeiro contexto, consideramos que cada caminho $P_i$  consiste de uma sequência de arestas consecutivas na grade ortogonal, que forma o caminho, e chamaremos essas de   {\it representações EPG}. Dessa forma, segue que dois caminhos intersectam-se se e somente se eles contém no mínimo uma aresta da grade em comum. Aos grafos que  correspondem às representações EPG denotaremos por {\it grafos EPG}. 
No segundo contexto, um caminho é visto como uma sequência de vértices consecutivos, e dois caminhos intersectam-se se eles contém um vértice comum. Analogamente aos anteriores esses são chamados de {\it representações VPG} e {\it grafos VPG}. 

Cada aresta possui uma direção associada na grade, a qual pode ser horizontal ou vertical. Uma  {\it dobra} no caminho é um par de arestas consecutivas que possuem direções distintas.  Um {\it segmento} de um caminho é uma sequência de arestas consecutivas do caminho, sem dobras. Dizemos que o caminho $P_i$ é um  $B_k$-{\it path} se ele contém $k$ dobras. Dizemos que $\cal {F}$ é uma família de $B_k$-paths, ou simplesmente  uma $B_k$-family, se cada caminho de $\cal {F}$ contém no máximo $k$ dobras. 

 Nesse capítulo, resolvemos completamente o problema de determinar ambos o número de Helly e o número de Helly forte, para ambos contextos de grafos $B_k$-EPG e $B_k$-VPG. Determinamos o número de Helly em grafos $B_k$-EPG e $B_k$-VPG, para cada valor de $k$.

Para grafos EPG, o número de Helly de $B_0$-families é bem conhecido e é igual a 2, uma vez que  grafos $B_0$-EPG coincidem com grafos de intervalo. Também é simples concluir que o número de Helly forte dos grafos $B_0$-EPG é também igual a 2. Para $k = 1$,   provamos que ambos o número de Helly e número de Helly forte da classe de $B_1$-families são iguais a 3. Para a classe de  $B_2$-families, provamos que esses dois parâmetros são iguais a 4. Além disso o número de Helly e número de Helly forte para $B_3$-families é igual a 8, e finalmente esses parâmetros são ilimitados para  $k \geq 4$. 

As for VPG graphs, it is simple to conclude that the Helly number of $B_0$-VPG graphs equals 2, and we prove that $B_1$-VPG have Helly number 4, $B_2$-VPG graphs have Helly number 6, $B_3$-VPG  have Helly number 12, while the Helly number for $B_4$-VPG graphs is again unbounded. 

Finally, the strong Helly number equals the Helly number of $B_k$-EPG graphs, for each $k$. Similarly, for $B_k$-VPG graphs. 

As for existing results, 
Golumbic, Lipshteyn  and Stern \cite{golumbic2009}  have already shown that the strong Helly number for $B_1$-EPG graphs equals 3, and for $B_1$-VPG graphs is equal to 4. employing a different proof technique. See  \cite{golumbic2019edge}, Theorem 11.13, below:
\begin{theorem}\label{thm:golumbic2019edge}{\cite{golumbic2019edge}}
Let $P$ be a collection of single bend paths on a grid. If every two paths in $P$ share at least one grid-edge, then $P$ has strong Helly number 3. Otherwise, $P$ has strong Helly number 4. 
\end{theorem}
No other results concerning the strong Helly number, or no results for the Helly number of $B_k$-EPG graphs seem to have been reported in the literature. As for other classes, Golumbic and Jamison have determined the strong Helly number of the intersection of edge paths of a tree \cite{golumbic1985}. Finally, Asinowski, Cohen, Golumbic, Limouzy, Lipshteyn and Stern have reported that the strong Helly number of $B_0$-VPG graphs equals 2 \cite{asinowski2011string}.  
Some related results are as follows. To decide  whether a given hypergraph is $k$-Helly can be done in polynomial time for fixed $k$, employing the characterization by Berge and Duchet \cite{bergeDuchet1975}. For arbitrary $k$, the problem is co-NP-complete \cite{dourado2009}. For the corresponding problems for strong $k$-Helly see \cite{dourado2008strong,dourado2009}.

The paper is organized as follows. Section 2, contains some preliminary propositions  and further notation. Section 3 describes  the results for the Helly number of $B_k$-EPG graphs, while Section 4 contains the results of this parameter for $B_k$-VPG graphs. The strong Helly number is considered in Section 5. Final remarks form the last section.

\section{Preliminaries}
The following theorem characterizes $h$-Helly families of subsets.


\begin{theorem}\label{thm:BD}(\cite{bergeDuchet1975}):
A family $\cal{F}$  of subsets of the universal set $U$ is $h$-Helly if and only if for every subset $U' \subseteq U$, $|U'|= h+1$, the subfamily $\cal{F'}$ of $\cal{F}$, formed by the subsets containing at least $h$ of the $h+1$ elements of $U'$, has a non-empty core. 
\end{theorem}

The next theorem is central to our results.

\begin{theorem}\label{thm:minimal}
Let ${\cal C}$ be a hereditary class of families ${\cal F}$ of subsets of the universal set $U$, whose Helly number $H({\cal C})$ equals $h$. Then there exists a family ${\cal F'} \in {\cal C}$ with exactly $h$ subsets, satisfying the following condition: 

For each subset $P_i \in \cal {F'}$, there is exactly one distinct element $u_i \in U$, such that \\
$$u_i \not \in P_i,$$ 
but $u_i$ is contained in all  subsets 
$$P_j \in {\cal F'} \setminus P_i.$$
\end{theorem}
 

Proof: 
Let ${\cal C}$ be a class of families ${\cal F}$ of subsets $P$, each subset formed by elements $u \in U$, such that the Helly number $H({\cal C})$ equals $h$. Then each family ${\cal F} \in {\cal C}$ satisfies $H({\cal F}) \leq h$. Consider a family ${\cal F'} \in {\cal C}$  whose Helly number is exactly $h$, and containing exactly $h$ subsets. Such a family must exist since ${\cal C}$ is hereditary. Since $H({\cal F'}) = h$, $\cal F'$ is $h$-intersecting, and therefore $(h-1)$-intersecting. Furthermore, ${\cal F'}$ is not $(h-1)$-Helly. Applying  Theorem \ref{thm:BD}, we conclude that there are $h$ elements $U' = \{u_1, \ldots, u_h\} \subset U$, such that each set of ${\cal F'}$ contains at least $h-1$ elements of $U'$. Since $H({\cal F'}) > h-1$, $core({\cal F'}) = \emptyset$ and therefore there is no common element among the sets of $\cal F'$. In particular, since each set $P_i \in {\cal F'}$ contains at least $h-1$ elements of $U'$, and $core(\cal F') = \emptyset$, we can choose $h$ subsets $P_i$, in which each of them misses a distinct element $u_i \in U'$ of $U'$. Then for each subset $P_i \in \cal F$, there exists some element $u_i \not \in P_i$, but $u_i \in P_j$, for all $P_j \in \cal F'$, $j \neq i$. \qed

Let $\cal{ F'}$ be as in the previous theorem. It is simple to conclude that the removal of any subset from $\cal {F'}$ turns it $(h-1)$-Helly.  Therefore we  call $\cal {F'}$ a {\it minimal non}-$(h-1)$-{\it Helly family}. Moreover, the element $u_i \not \in P_i$, contained in all subsets $P_j \in {\cal{F'}} \setminus P_i$, except $P_i$, is the {\it $h$-non-representative} of $P_i$.  

We will employ the above minimal families of subsets, applied to $B_k$-paths in a grid. Note that $B_k$-paths in a grid form a hereditary class. 

\section{The Helly Number of $B_k$-EPG Graphs}\label{sec:Helly-number}

In this section, we determine the Helly number of the classes of $B_1$-EPG, $B_2$-EPG and $B_3$-EPG graphs, and show that for $B_k$-EPG graphs, $k \geq 4$, the Helly number is unbounded. We prove the following results.

\begin{theorem}\label{thm:Helly-EPG}
The Helly number of $B_k$-EPG graphs satisfy:
\begin{enumerate}[nosep,label=\emph{(\roman*)}]
\item  $H(B_1$-EPG) = 3 
\item $H(B_2$-EPG)  = 4 
\item $H(B_3$-EPG)  = 8 
\item $H(B_k$-EPG) is unbounded, for 
$k \geq 4$.
\end{enumerate}

\end{theorem}

The proof consists in determining tight lower and upper bounds, as shown in the next two subsections. 

\subsection{Lower Bounds}

First, we describe lower bounds for this parameter,  as a function of the number $k$ of bends.

\begin{lema}\label{claim:lower-Bk-EPG} 
The following are lower bounds for $B_k$-EPG graphs.
\begin{enumerate}[nosep,label=\emph{(\roman*)}]
\item   $H(B_1$-$EPG) \geq 3$ 
\item $H(B_2$-$EPG) \geq 4$ 
\item $H(B_3$-$EPG) \geq 8$ 
\item $H(B_k$-$EPG )$ is unbounded for $k \geq 4$.
\end{enumerate}
\end{lema}

\proof:

For each such value of $k$, we exhibit a $B_k$-family of edge paths, having the required number of bends, and whose Helly number is at least the corresponding stated value. We refer to the pair of coordinates of grid points, in order to describe the paths.

\begin{figure}[!h]
\begin{center}
% \begin{tikzpicture}[line cap=round,line join=round,>=triangle 45,x=3.7mm,y=3.7mm]
% \draw [color=cqcqcq,, xstep=0.37cm,ystep=0.37cm] (-7,-1.4) grid (22.16,6.8);
% \clip(-5.7,-2.5) rectangle (25,9.6);
% \draw (-5,-1.3) node[anchor=north west] {(a)};
% \draw (2.5,-1.3) node[anchor=north west] {(b)};
% \draw (14.5,-1.3) node[anchor=north west] {(c)};
% \draw [line width=2pt] (4,0)-- (4,2)-- (6,2)-- (6,0);
% \draw [line width=2pt] (3,2)-- (1,2)-- (1,0)-- (3,0);
% %\draw (-0.3,3.7) node[anchor=north west] {0};
% %\draw (-0.3,5.1) node[anchor=north west] {1};
% %\draw (-0.3,6.1) node[anchor=north west] {2};
% \draw [line width=2pt] (1,5)-- (3,5)-- (3,3)-- (1,3);
% \draw [line width=2pt] (4,5)-- (4,3)-- (6,3)-- (6,5);
% %\draw (8.7,0.7) node[anchor=north west] {0};
% %\draw (8.7,2.1) node[anchor=north west] {1};
% %\draw (8.7,3.1) node[anchor=north west] {2};
% %\draw (8.7,3.7) node[anchor=north west] {0};
% %\draw (8.7,5.1) node[anchor=north west] {1};
% %\draw (8.7,6.1) node[anchor=north west] {2};
% \draw [line width=2pt] (10,5)-- (12,5)-- (12,3)-- (10,3)-- (10,4);
% \draw [line width=2pt] (14,5)-- (13,5)-- (13,3)-- (15,3)-- (15,5);
% \draw [line width=2pt] (19,3)-- (19,5)-- (21,5)-- (21,3)-- (20,3);
% \draw [line width=2pt] (18,4)-- (18,5)-- (16,5)-- (16,3)-- (18,3);
% \draw [line width=2pt] (10,1)-- (10,2)-- (12,2)-- (12,0)-- (10,0);
% \draw [line width=2pt] (13,2)-- (13,0)-- (15,0)-- (15,2)-- (14,2);
% \draw [line width=2pt] (20,0)-- (19,0)-- (19,2)-- (21,2)-- (21,0);
% \draw [line width=2pt] (18,2)-- (16,2)-- (16,0)-- (18,0)-- (18,1);
% \draw [line width=2pt] (-6,2.25)-- (-4.25,2.25)-- (-4.25,4);
% \draw [line width=2pt] (-3.75,4)-- (-3.75,2.25)-- (-2,2.25);
% \draw [line width=2pt] (-6,1.75)-- (-2,1.75);
% \end{tikzpicture}
\includegraphics[width=12cm]{./img/b1epgSub.pdf}
\end{center}
\caption{Minimal non-Helly sub-families for the $B_1$, $B_2$ and $B_3$ -families.}
\end{figure}

For $k=1$, let $\cal{F}$ be a family of three 1-bend paths that pairwise intersect but which have no common edge, as depicted in Figure $1 (a)$. 
%For $k=1$, let $\cal{F}$ be the family %of three 1-bend paths, $P_1: (0,0),(0,1),(1,1)$; $P_2: (1,1), (1,0),(0.2)$; and $P_3: (0,0),(0,2)$.  See Figure   $1a$. 
Then $\cal{F}$ is a  2-intersecting $B_1$-EPG family of three paths, having an empty core. Therefore, $H(B_1$-EPG$) \geq 3$. 
Furthermore, by  removing of any of the paths the core of  $\cal{F}$ turns non-empty. Therefore $\cal{F}$ is a minimal non 2-Helly family of paths.

Let $S$ be the 4-cycle  formed by the four two-edge segments, with bends at the grid points $(0,0),(0,2),(2,2),(2,0)$, respectively.
For $k= 2$,  consider $\cal{F}$ to be the family of four 2-bend paths
formed when we remove exactly one of the two-edge segments that form the 4-cycle,  as depicted in Figure $1 (b)$.
It follows that $\cal{F}$ is 3-intersecting and its paths have no common edge.  Hence $H(B_2$-EPG)$ \geq 4$.

For $k=3$, consider the family $\cal{F}$ of eight  3-bend paths, respectively, which can be obtained from $S$. The 4-cycle $S$ contains exactly 8 grid edges. The  family $\cal{F}$ consists  of the 8 paths $P_i$, $1 \leq i \leq 8$, obtained by removing from $S$, exactly one of these distinct 8 edges, as depicted in Figure $1 (c)$. Consequently, $\cal{F}$ is 7-intersecting, but $core({\cal{F}}) = \emptyset$. Therefore, $H(B_3$-EPG)$\geq 8$.

Finally, for $k = 4$, let $\cal{F}$ be the family of $n$ $B_4$-paths $P_i$, described as follows: 

\begin{itemize}
    \item $P_1$ is formed by the segments: \\ $(0,0),(0,1),(1,1),(1,0),(n,0)$; 

     \item for $2 \leq i \leq n-1$, $P_i$ contains the segments: \\
     $(0,0),(0,i-1),(i-1,1),(i,1),(i,0),(n,0)$;
     
     \item  $P_n$ is formed by the segments: \\ $(0,0),(n-1,0),(n-1,1),(n-1,0).$
     
\end{itemize}     

Observe that $\cal{F}$ is $(n-1)$-intersecting, while $core({\cal{F}})=\emptyset$. See Figure 2. Therefore $H(B_4$-EPG) is unbounded. Clearly the same holds for $k >4$. \qed  



\begin{figure}[!h]
\begin{center}
% \begin{tikzpicture}[line cap=round,line join=round,>=triangle 45,x=3.7mm,y=3.7mm]
% \draw [color=cqcqcq,, xstep=0.74cm,ystep=0.74cm] (-7,-1.0) grid (22.8,14.8);
% \clip(-6.7,-1.8) rectangle (27,14.6);
% \draw [line width=2pt] (-6,0)-- (-6,2)-- (-4,2)-- (-4,0)-- (22,0);

% \draw [line width=2pt] (-6,4)-- (-4,4)-- (-4,6)-- (-2,6)-- (-2,4)-- (22,4);

% %\draw [line width=2pt] (-6,6)-- (-2,6)-- (-2,8)-- (0,8)-- (-0,6)-- (22,6);

% \draw [line width=2pt] (-6,8)-- (10,8)-- (10,10)-- (12,10)-- (12,8)-- (22,8);

% \draw [line width=2pt] (-6,12)-- (20,12)-- (20,14)-- (22,14)-- (22,12);


% \draw (6.5,6.5) node[anchor=north west] {.};
% \draw (6.5,7) node[anchor=north west] {.};
% \draw (6.5,7.5) node[anchor=north west] {.};

% \draw (6.5,10.5) node[anchor=north west] {.};
% \draw (6.5,11) node[anchor=north west] {.};
% \draw (6.5,11.5) node[anchor=north west] {.};

% \draw (-5.65,0.4) node[anchor=north west] {1};
% \draw (-3.65,3.4) node[anchor=north west] {2};

% %\draw (-1.65,6.4) node[anchor=north west] {3};

% \draw (10.35,8.4) node[anchor=north west] {i};

% \draw (20.35,12.4) node[anchor=north west] {n};

% \end{tikzpicture}
\includegraphics[width=12.5cm]{./img/b4epg.pdf}
\end{center}
\caption{$B_4$ has an unlimited Helly number.}
\end{figure}

Next, we consider finding  upper bounds for $H(B_k$-EPG) graphs.

\subsection{Upper Bounds}\label{subsec-upper}

In order to obtain  tight upper bounds for the Helly number, in terms of the number of bends, we introduce below, more notation and lemmas.

Say that a set of edges of a grid is {\it co-linear} if all edges of the set belong to a same line of the grid, horizontal or vertical. The set of edges is called {\it parallel} if all its edges lie on parallel lines of the grid, but no two of them are co-linear.  


\begin{lemma}
\label{lemma:3colin}
Let $\cal {F}$ be a minimal non-$(h-1)$-Helly family of paths on a grid  containing  three co-linear non representative edges. Then $\cal{F}$ must contain paths with at least four bends.
\end{lemma}

\proof
Let $u_i$ be the middle one of the three co-linear non representative edges. It corresponds to the  path $P_i$ of $\cal {F'}$, not containing $u_i$.
Then $P_i$ must go through  the other two edges but it cannot include the middle edge. Therefore path $P_i$ must leave the common line of the grid, containing those three representatives edges, and return to that same line, thus requiring at least four bends.
\qed


\begin{lemma}
\label{lemma:3par}
Let $\cal{F}$ be a minimal non-$(h-1)$-Helly family of paths on a grid, containing three parallel edges, and having  Helly number $H(\cal{F})$   $\geq 4$. Then $\cal{F}$ must contain paths with at least four bends. 
\end{lemma}

\proof
Since $H(\cal{F}) $ $\geq 4$ and $\cal{F}$ is a minimal $(h-1)$-family, it follows that $\cal{F}$ must contain at least four paths, $P_1,P_2,P_3,P_4$. Without loss of generality, let $u_1,u_2,u_3$ be the non-representatives edges of the paths $P_1,P_2,P_3$ which are parallel. Then $P_4$ must go through all the three parallel non-representative edges $u_1,u_2,u_3$, thus requiring at least four bends. 
\qed


\begin{lemma} \label{lemma:Lwit}
Let $\cal{F}$ be a minimal non-$(h-1)$-Helly family of paths on a grid with  Helly number $H({\cal F}) \geq 4$. If ${\cal(F)}$ contains  three non-representative edges that  lie on a common $B_1$-path $P_1$ of $\cal{F}$, then $\cal {F}$ must have some path with at least three bends. \end{lemma}

\proof
Since $\cal{F}$ is a minimal $(h-1)$-family having Helly number $\geq 4$, it contains at least four paths. Without loss of generality, let  $u_1, u_2, u_3$ be the three non-representative edges contained in $P_4$ and such that  $u_2$ lies between $u_1$ and $u_3$ in $P_4$. Then path $P_2$ must contain $u_1$ and $u_3$, but avoid $u_2$, thus requiring at least three bends.  
\qed

The following are actual  tight upper bounds for the Helly numbers of edge $B_k$-paths, for $k = 1,2,3$.

\begin{lema}\label{claim:upper-B1}
$H(B_1$-$EPG) \leq 3.$
\end{lema}
 
\proof
Assume by contradiction that the Helly number of $B_1$-EPG paths is $h > 3$. In this case, consider a minimal non-$(h-1)$-Helly family of $\cal F$ of $B_1$-paths. Then $\cal F$ contains at least $h$ paths.  
Any path $P_1 \in \cal{F}$ must contain $h-1$ non-representative edges  corresponding to the $h-1$ distinct paths of $\cal F$ other than $P_1$. Since $h-1 \geq 3$, $P_1$ contains at least three distinct non-representative edges $u_2, u_3, u_4 \in P_i$, with $u_3$ lying  between $u_2$ and $u_4$ in the path.   If $P_1$ has no bends, $u_2,u_3,u_4$ are co-linear. By Lemma~\ref{lemma:3colin},  path $P_3 \in \cal{F}$ must contain at least four bends. If $P_1$ has exactly one bend, it follows from Lemma~\ref{lemma:Lwit} that $P_3$ has three bends. In any situation, a contradiction arises, implying that $H({\cal F}) \leq 3$.
\qed

\begin{lema}\label{claim:upper-B2}
$H(B_2$-EPG$) \leq 4.$
\end{lema}

\proof
Assume by contradiction that the Helly number of  $B_2$-EPG families of paths is $h > 4$. In this case, consider a minimal non-$(h-1)$-Helly family $\cal F$ of $B_2$-EPG paths. The family  $\cal F$ must contain at least $h \geq 5$ distinct paths, each of them corresponding to a distinct non-representative  edge. Choose arbitrarily 5 of these non-representative edges.

By Lemmas ~\ref{lemma:3colin} and ~\ref{lemma:3par} any three of these chosen edges can  neither be co-linear nor parallel. Therefore at least one of the 5 chosen non-representative edges must be in a direction that is different from the majority of the chosen edges. Call the direction of this edge vertical, and the direction of the majority of the chosen edges horizontal. Consider a path $P_1$ from the family $\cal F$ that goes through this vertical edge. 
The path $P_1$ contains at least four of the chosen non-representative edges, at least one of which is vertical. Since $P_1$ has at most 2 bends then it must have at most three segments. Since we have three segments and four non-representative edges which $P_1$ must contain, by the pigeon hole principle, one of these segments must have two non-representative edges. If this pair of edges are in a horizontal segment of $P_1$, then such  pair of edges, along with the vertical edge are in two consecutive path segments, forming a $B_1$-subpath in $\cal F$. Then Lemma~\ref{lemma:Lwit} implies that some path of $\cal F$ must have at least three bends.   Otherwise, the two edges are vertical. But  the others must be horizontal, and again we have at least three edges in a pair of consecutive segments forming a subpath in $\cal F$ having one bend. Again,  Lemma ~\ref{lemma:Lwit} implies that some path has at least three bends.
\qed

\begin{lema}\label{claim:upper-B3}
$H(B_3$-EPG$) \leq 8.$
\end{lema}

\proof
Assume by contradiction that the Helly  number of  $B_3$-EPG paths is $h > 8$. In this case, consider a minimal non-$(h-1)$-Helly family $\cal F$ of $B_3$-EPG paths. Then $\cal F$ contains at least $h$  distinct non-representative edges,  corresponding to $h$ distinct subsets.  By Lemma~\ref{lemma:3par} since we can have at most three bends in any path, then these $h$  non-representative edges must lie in at most two vertical and two horizontal lines of the grid. Therefore one of these four possible lines must contain at least three distinct non-representative edges. By Lemma~\ref{lemma:3colin},  that would imply the existence of a path with four bends.\qed

This completes the proof of Theorem \ref{thm:Helly-EPG}. 


\section{Helly number of $B_k$-VPG Graphs}

In this section, we determine the Helly number of $B_k$-VPG graphs. We prove the following results.
\begin{theorem}\label{thm:Bk-VPG}
The Helly numbers for $B_k$-VPG graphs satisfy:
\begin{enumerate}
\item $H(B_1$-VPG) = 4
\item $H(B_2$-VPG) = 6
\item $H(B_3$-VPG) = 12
\item $H(B_4$-VPG) is unbounded.
\end{enumerate}
\end{theorem}

Again, we prove the theorem by showing tight lower and upper bounds.

\subsection{Lower Bounds}

First, we describe lower bounds.

Figure \ref{VPG:lower-B1} shows a set of 4 $B_1$-paths of a graph $G$, in a $2 \times 2$ grid, such that each path covers 3  vertices of $G$, and avoids exactly one of the  vertices. 

\begin{figure}[!h]
    \centering
    \includegraphics[width=3cm]{./img/lower-bound-B1-VPG.pdf}
    \caption{Lower bound for $B_1$-VPG graphs}
    \label{VPG:lower-B1}
\end{figure}

Figure \ref{VPG:lower-B2} shows a set of 6 $B_2$-paths of a graph $G$, in a $2 \times 3$ grid, such that each path covers 5  vertices of $G$, and avoids exactly one. 


\begin{figure}[!h]
    \centering
    \includegraphics[width=8cm]{./img/lower-bound-B2-VPG.pdf}
    \caption{Lower bound for $B_2$-$VPG$ graphs}
    \label{VPG:lower-B2}
\end{figure}


Figure \ref{VPG:lower-B3} shows 12 $B_3$-paths of a graph $G$, in a grid, of perimeter 12, such that each path covers 11  vertices of $G$,  avoiding one of them. 

\begin{figure}[!h]
    \centering
    \includegraphics[width=12cm]{./img/lower-bound-B3-VPG.pdf}
    \caption{Lower bound for $B_3$-VPG graphs}
    \label{VPG:lower-B3}
\end{figure}

Figure \ref{VPG:lower-B4} shows a set of $n$ $B_4$-paths of a $n$-vertex graph $G$, in a grid having perimeter $n$,  such that each path covers $n-1$  vertices of $G$, avoiding one of them. 

\begin{figure}[!h]
    \centering
    \includegraphics[width=12cm]{./img/lower-bound-B4-VPG.pdf}
    \caption{Lower bound for $B_4$-VPG graphs}
    \label{VPG:lower-B4}
\end{figure}

Applying Theorem \ref{thm:minimal}, we can then conclude that the number of vertices of each of the above described graphs are lower bounds for the corresponding class. Then, we can claim the following bounds.

\begin{lema}\label{claim:VPG-lower}
The following are lower bounds for $H(B_k$-VPG) graphs.
\begin{enumerate}
\item $H(B_1$-VPG) $\geq 4$
\item $H(B_2$-VPG) $\geq 6$
\item $H(B_3$-VPG) $\geq 12$
\item $H(B_4$-VPG) is unbounded.
\end{enumerate}
\end{lema}

\subsection{Upper Bounds}

Next, we provide upper bounds for the Helly number of $B_k$-VPG graphs. The following lemmas are employed.

\begin{lemma}\label{column-sizes}
Let $\cal F$ be a minimal non-$(h-1)$-Helly family of paths, for some $h$, containing $k \in \{3,4,5\}$ distinct co-linear non-representative points of the grid. Then $\cal F$ contains a path having at least $k-1$ bends.
\end{lemma}

\proof For $k \in \{3,5\}$, the path avoiding the middle point has at least $k-1$ bends; while for $k = 4$ the path avoiding one of the middle points also has this same property.
\qed

\begin{lemma}\label{column-number}
Let $\cal F$ be a minimal non-$(h-1)$-Helly family of paths, on a grid containing $k < h$ distinct pairwise non-co-linear non-representative points. Then $\cal F$ must contain a path with at least $k-1$ bends.
\end{lemma}   

\proof Since $k < h$, $\cal F$ must contain a path that visits all such $k$ pairwise non-co-linear points. Such a path requires at least one bend, between two consecutive non-co-linear points. Therefore $\cal F$ contains a path with at least $k-1$ bends. \qed \\

We also employ some additional concepts and notation, below described.

Let $\cal F$ be a minimal non-$(h-1)$-Helly family of $B_{k-1}$-paths on a grid $Q$. By Theorem \ref{thm:minimal},  we can choose $h$ paths $P_i \in {\cal F}$, each of them associated to a distinct non-representative grid point $p_i$, such that $P_i$ avoids $p_i$, but contains all the other $h-1$ distinct non-representative points $p_j \in P_J$, for each   $j \neq i$. Denote by $P_N$, $|P_N|=h$, the subset of grid  points of  $Q$, restricted to the chosen set of distinct  non-representative points $p_i$. By Lemmas \ref{column-sizes} and \ref{column-number}, the grid points of $P_N$ are contained in at most $k$ columns (lines), and each column (line) contains at most $k$ points of $P_N$. Consequently, the cardinalities of the points of $P_N$, contained in the columns (lines) of $Q$,  form a partition of the integer $h$, into at most $k$ parts, such that each part is at most $k$. Call such a partition as a {\it feasible  partition  of $h$, relative to $P_N$}. Therefore, each non-representative point $p_i \in P_N$ contributes with one unit to some part of the partition, which is then referred to,   as the part of the partition {\it corresponding} to $p_i$.    

The following lemma describes sufficient conditions for an integer $h$ to be an upper bound of the Helly number.

\begin{lemma}\label{upper-bound} Let $\cal F$ be a minimal non-$(h-1)$-Helly family of $B_{k-1}$-paths on a grid $Q$, and $P_N$ the set of non-representative points of $Q$. Let $k,h$ be integers, $1 \leq k \leq 3$ and $k < h$. The following conditions imply $H(B_k$-VPG) $\leq h$  
\begin{itemize}
    \item[(i)] there is no feasible partition of $h+1$, relative to $P_N$, or 
    \item[(ii)] for any possible feasible partition, and for any arrangement of the grid points of $P_N$ in $Q$, there is some non-representative point $p_i \in P_N$, such that  no path exists  in $Q$, having at most $k$ bends, containing all points of $P_N$, except $p_i$.    
\end{itemize}
\end{lemma}
{\it Proof}: The proof of (i) follows from Lemmas \ref{column-sizes} and \ref{column-number}, while the proof of (ii) is a consequence of Theorem \ref{thm:minimal}.  \qed \\

The following are upper bounds for the Helly number of $B_k$-VPG graphs, for each $k$, $1 \leq k \leq 3$, obtained  by applying Lemma \ref{upper-bound}.      
 
\begin{lema}\label{claim:upper-B1-VPG}
$H(B_1$-VPG) $\leq  4$.
\end{lema}

\proof There is no partition of the integer 5, into 2 parts, in which each part is at most 2. Consequently, the result follows from Lemma \ref{upper-bound} (i). \qed

\begin{lema}\label{claim:upper-B2-VPG}
$H(B_2$-VPG)  $\leq  6$.
\end{lema}

\proof Assume the contrary. Then $H(B_2$-VPG) $\geq  7$, let $\cal F$ be a minimal non-6-Helly family of $B_2$-paths, and  $P_N$ be the set of non-representative points of $\cal F$ in $Q$. There are two possible partitions of the integer 7, in three parts, each of them of size at most 3, namely $(3,3,1)$ and $(3,2,2)$. In any of these cases,  it is always possible to choose some point  $p_i \in P_N$, belonging  to a part of the partition of size 3, such that a path in $\cal F$  which  avoids $p_i$ and covers the other 6 non-representative points, must contain at least 3 bends.  Then by Lemma \ref{upper-bound}, indeed $H(B_2$-VPG)  $\leq  6$. \qed


\begin{lema}\label{claim:upper-B3-VPG}
$H(B_3$-VPG) $\leq  12$.
\end{lema}

\proof Assume the contrary, $H(B_3$-VPG) $\geq  12$. Let $\cal F$ be a minimal non-12-Helly family of $B_3$-paths, and  $P_N$ be the set of non-representative points of $\cal F$ in $Q$. There are three possible partitions of the integer 13, into four parts, each of them of size at most 4, namely $(4,4,4,1)$, $(4,4,3,2)$ and $(4,3,3,3)$. In this case, choose $p_i \in P_N$ to be a non-representative point, corresponding to a part of size $4$ of the partition.  The path of ${\cal F}$, which avoids $p_i$ must cover the other 12 non-representative points. These points are located in 4 distinct columns, of cardinalites 4,4,3,1, 4,3,3,2, or 3,3,3,3, considering the 3 possible partitions, respectively. Such a path must contain at least 4 bends, a contradiction. Then by Lemma \ref{upper-bound}, $H(B_3$-VPG) $\leq  12$.    \qed  

From the lower and upper bounds described in the previous subsections, we obtain the results for the Helly numbers of $B_k$-VPG graphs, completing the proof of Theorem \ref{thm:Bk-VPG}.

\section{Strong Helly Number} 

In this section, first  we consider determining the strong Helly number of $B_k$-EPG graphs.

We start by describing a theorem similar to Theorem \ref{thm:minimal}.

\begin{theorem}\label{thm:minimal-strong}

Let ${\cal C}$ be a hereditary class of families $\cal F$ of subsets of the universal set $U$, whose strong Helly number $sH({\cal C})$ equals $h$. Then there exists a family ${\cal F'} \in {\cal C}$ with exactly $h$ subsets satisfying the following condition: 

For each subset $P_i \in \cal {F'}$, there is exactly one distinct element $u_i \in U$, such that \\
$$u_i \not \in P_i,$$ 
but $u_i$ is contained in all  subsets 
$$P_j \in {\cal F'} \setminus P_i.$$
\end{theorem}

Proof: The strong Helly number of ${\cal C}$ is $h$ and not $h - 1$, so that  there must exist some family ${\cal F} \in {\cal C}$ whose strong Helly number is exactly $h$, that is, $\cal F$  contains $h$ subsets $P_i$ whose intersection equals  core($\cal F'$) but is such that no  $h-1$ of its subsets have the same intersection. In particular, let $\cal F'$ be the family containing exactly the $h$ subsets $P_i$ described above. Such a family must exist, since $\cal C$ is hereditary. Then each $P_i$ does not contain at least one element $u_i$ in the intersection of the remaining $h-1$ subsets $P_j$, $j \ne i$, 
since the intersection of these $h-1$ subsets must not be equal to the core($\cal F'$).  \qed

Again, if we consider the family $\cal F'$ described in the theorem above it is simple to conclude that the removal of any subset from $\cal {F'}$ turns it $(h-1)$-strong Helly.  Then call $\cal {F'}$ a {\it minimal} non-$(h-1)$-strong Helly family. Moreover, the element $u_i \not \in P_i$, contained in all subsets $P_j \in {\cal{F'}} \setminus P_i$, except $P_i$, is the {\it $h$ non-representative} of $P_i$.  

As before, we  employ the above minimal families of subsets, applied to paths in a grid.

In fact, we prove that the strong Helly number of $B_k$-EPG graphs coincide with the Helly number, for each corresponding value of $k$. Similarly, for $B_k$-VPG graphs. For $k=0$, it is simple to show that if a set of intervals $\cal I$ in a line pairwise intersect then there exist two intervals of $\cal I$, whose intersection equals the intersection of all intervals of $\cal I$. Consequently the $k$-strong Helly number of $B_0$-EPG graphs equals 2. 
Similarly, for $B_0$-VPG graphs. 
Recall that the strong Helly number is at least equal to the Helly number of a family, so that the lower bounds presented in Claim~\ref{claim:lower-Bk-EPG} also hold for the strong Helly number. The proofs for the strong Helly numbers for $k \geq 1$ are similar to those described in Section \ref{sec:Helly-number}.  



\section{Concluding Remarks}
We have determined the Helly number and strong Helly number of $B_k$-EPG graphs and $B_k$-VPG graphs, for $k \geq 0$. 

Table \ref{tab:Helly-Strong-Helly} summarizes the results obtained.
 
\Large 

\begin{table}
    \centering
    \begin{tabular}{c|c|c}
    \cline{1-3} $k$  & $B_k$-EPG & $B_k$-VPG \\
    \cline{1-3} 0 & 2 & 2 \\
    \cline{1-3} 1 & 3 & 4 \\
    \cline{1-3} 2 & 4 & 6 \\
    \cline{1-3} 3 & 8 & 12 \\
    \cline{1-3} $\geq 4$ & unbounded & unbounded \\
    \cline{1-3} 
    \end{tabular}
    \caption{Helly and Strong Helly Numbers for $k$-EPG and $k$-VPG Graphs}
    \label{tab:Helly-Strong-Helly}
\end{table}

\normalsize


We leave two questions to be investigated, in relation to the presented results.

\begin{enumerate}
\item Given a {\it specific}  EPG or VPG graph, the question is to formulate an algorithm to determine its Helly and strong Helly numbers. See \cite{dourado2008improved}, for instance, for such algorithms, applied to general graphs. 

\item The values of the Helly and strong Helly numbers which were determined in this paper coincided in all cases. Clearly, in general, this is not the case. We leave as an open question, to find the conditions for such an equality to occur. 
\end{enumerate}


% \begin{thebibliography}{99}
% \bibitem{asinowski2011string}
% A. Asinowski, E. Cohen, M. C. Golumbic, V. Limouzy, M. Lipshteyn and M. Stern. Electronic Notes in Discrete Mathematics 37 (2011), pp. 141-146. 

% \bibitem{asinowski2012}  A. Asinowski, E. Cohen, M. C. Golumbic, V. Limouzy, M. Lipshteyn, and M. Stern,
% Vertex intersection graphs of paths on a grid, Journal of Graph Algorithms and
% Applications, 16 (2012) pp. 129-150.

% \bibitem{bergeDuchet1975}
% C. Berge and P. Duchet. A generalization of Gilmore’s theorem, {\emph in} M. Fiedler,
% editor, Recent Advances in Graph Theory, Acad. Praha, Prague, 1975, pp. 49-55

% \bibitem{duchet1978propriete}
% P. Duchet. Propriet\'e de Helly et probl\`emes de repr\'esentations. In Colloquium
% International CNRS 260, Probl\'emes Combinatoires et Th\'eorie de Graphs, Orsay, France, 1976, pp. 117-118

%  \bibitem{dourado2008improved}
%  M. C. Dourado, M. C. Lin, F. Protti, and J. L. Szwarcfiter. Improved algorithms
%  for recognizing $p$-Helly and hereditary $p$-Helly hypergraphs. Information Processing Letters 108  (2008), pp. 257-250.

% \bibitem{dourado2008strong}
% M. C. Dourado, F. Protti, and J. L. Szwarcfiter. On the strong $p$-Helly property.
% Discrete Applied Mathematics, 156 (2008), pp. 1053–1057

% \bibitem{dourado2009}
% M. C. Dourado, F. Protti and J. L. Szwarcfiter,
% Complexity aspects of the Helly
% property: graphs and hypergraphs. Electronic Journal on Combinatorics, Dynamic Surveys 17, 2009 

% \bibitem{golumbic1985}
% M. C. Golumbic and R. E. Jamison,
% The edge intersection graphs of paths in a tree,
% Journal of Combinatorial Theory B 38 (1985), pp. 8-22.

% \bibitem{golumbic2009}
% M. C. Golumbic, M. Lipshteyn and  M. Stern,
% Edge intersection graphs of single bend paths on a grid, Networks 54 (2009), pp. 130-138.

% \bibitem{golumbic2013}
% M. C. Golumbic, M. Lipshteyn and  M. Stern,
% Single bend paths on a grid have strong Helly number 4,
% Networks (2013), 161-163

%\bibitem{}
% M. C. Golumbic and G. Morgenstern,
% Edge intersection graphs of paths
% on a grid, {\emph in} ``50 Years of Combinatorics, Graph Theory and Computing'', F.~Chung, R.~Graham, F.~Hoffman, L.~Hogben, R.~Mullin, D.~West, eds, CRC Press, 2019, pp. 193-209. 